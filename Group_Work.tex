% Options for packages loaded elsewhere
% Options for packages loaded elsewhere
\PassOptionsToPackage{unicode}{hyperref}
\PassOptionsToPackage{hyphens}{url}
\PassOptionsToPackage{dvipsnames,svgnames,x11names}{xcolor}
%
\documentclass[
  a4paper,
  DIV=11,
  numbers=noendperiod]{scrartcl}
\usepackage{xcolor}
\usepackage[top=25mm,left=25mm,right=25mm,bottom=25mm,heightrounded]{geometry}
\usepackage{amsmath,amssymb}
\setcounter{secnumdepth}{-\maxdimen} % remove section numbering
\usepackage{iftex}
\ifPDFTeX
  \usepackage[T1]{fontenc}
  \usepackage[utf8]{inputenc}
  \usepackage{textcomp} % provide euro and other symbols
\else % if luatex or xetex
  \usepackage{unicode-math} % this also loads fontspec
  \defaultfontfeatures{Scale=MatchLowercase}
  \defaultfontfeatures[\rmfamily]{Ligatures=TeX,Scale=1}
\fi
\usepackage{lmodern}
\ifPDFTeX\else
  % xetex/luatex font selection
  \setmainfont[]{Source Serif 4}
  \setsansfont[]{Source Sans 3}
  \setmonofont[]{Source Code Pro}
\fi
% Use upquote if available, for straight quotes in verbatim environments
\IfFileExists{upquote.sty}{\usepackage{upquote}}{}
\IfFileExists{microtype.sty}{% use microtype if available
  \usepackage[]{microtype}
  \UseMicrotypeSet[protrusion]{basicmath} % disable protrusion for tt fonts
}{}
\makeatletter
\@ifundefined{KOMAClassName}{% if non-KOMA class
  \IfFileExists{parskip.sty}{%
    \usepackage{parskip}
  }{% else
    \setlength{\parindent}{0pt}
    \setlength{\parskip}{6pt plus 2pt minus 1pt}}
}{% if KOMA class
  \KOMAoptions{parskip=half}}
\makeatother
% Make \paragraph and \subparagraph free-standing
\makeatletter
\ifx\paragraph\undefined\else
  \let\oldparagraph\paragraph
  \renewcommand{\paragraph}{
    \@ifstar
      \xxxParagraphStar
      \xxxParagraphNoStar
  }
  \newcommand{\xxxParagraphStar}[1]{\oldparagraph*{#1}\mbox{}}
  \newcommand{\xxxParagraphNoStar}[1]{\oldparagraph{#1}\mbox{}}
\fi
\ifx\subparagraph\undefined\else
  \let\oldsubparagraph\subparagraph
  \renewcommand{\subparagraph}{
    \@ifstar
      \xxxSubParagraphStar
      \xxxSubParagraphNoStar
  }
  \newcommand{\xxxSubParagraphStar}[1]{\oldsubparagraph*{#1}\mbox{}}
  \newcommand{\xxxSubParagraphNoStar}[1]{\oldsubparagraph{#1}\mbox{}}
\fi
\makeatother


\usepackage{longtable,booktabs,array}
\usepackage{calc} % for calculating minipage widths
% Correct order of tables after \paragraph or \subparagraph
\usepackage{etoolbox}
\makeatletter
\patchcmd\longtable{\par}{\if@noskipsec\mbox{}\fi\par}{}{}
\makeatother
% Allow footnotes in longtable head/foot
\IfFileExists{footnotehyper.sty}{\usepackage{footnotehyper}}{\usepackage{footnote}}
\makesavenoteenv{longtable}
\usepackage{graphicx}
\makeatletter
\newsavebox\pandoc@box
\newcommand*\pandocbounded[1]{% scales image to fit in text height/width
  \sbox\pandoc@box{#1}%
  \Gscale@div\@tempa{\textheight}{\dimexpr\ht\pandoc@box+\dp\pandoc@box\relax}%
  \Gscale@div\@tempb{\linewidth}{\wd\pandoc@box}%
  \ifdim\@tempb\p@<\@tempa\p@\let\@tempa\@tempb\fi% select the smaller of both
  \ifdim\@tempa\p@<\p@\scalebox{\@tempa}{\usebox\pandoc@box}%
  \else\usebox{\pandoc@box}%
  \fi%
}
% Set default figure placement to htbp
\def\fps@figure{htbp}
\makeatother


% definitions for citeproc citations
\NewDocumentCommand\citeproctext{}{}
\NewDocumentCommand\citeproc{mm}{%
  \begingroup\def\citeproctext{#2}\cite{#1}\endgroup}
\makeatletter
 % allow citations to break across lines
 \let\@cite@ofmt\@firstofone
 % avoid brackets around text for \cite:
 \def\@biblabel#1{}
 \def\@cite#1#2{{#1\if@tempswa , #2\fi}}
\makeatother
\newlength{\cslhangindent}
\setlength{\cslhangindent}{1.5em}
\newlength{\csllabelwidth}
\setlength{\csllabelwidth}{3em}
\newenvironment{CSLReferences}[2] % #1 hanging-indent, #2 entry-spacing
 {\begin{list}{}{%
  \setlength{\itemindent}{0pt}
  \setlength{\leftmargin}{0pt}
  \setlength{\parsep}{0pt}
  % turn on hanging indent if param 1 is 1
  \ifodd #1
   \setlength{\leftmargin}{\cslhangindent}
   \setlength{\itemindent}{-1\cslhangindent}
  \fi
  % set entry spacing
  \setlength{\itemsep}{#2\baselineskip}}}
 {\end{list}}
\usepackage{calc}
\newcommand{\CSLBlock}[1]{\hfill\break\parbox[t]{\linewidth}{\strut\ignorespaces#1\strut}}
\newcommand{\CSLLeftMargin}[1]{\parbox[t]{\csllabelwidth}{\strut#1\strut}}
\newcommand{\CSLRightInline}[1]{\parbox[t]{\linewidth - \csllabelwidth}{\strut#1\strut}}
\newcommand{\CSLIndent}[1]{\hspace{\cslhangindent}#1}



\setlength{\emergencystretch}{3em} % prevent overfull lines

\providecommand{\tightlist}{%
  \setlength{\itemsep}{0pt}\setlength{\parskip}{0pt}}



 


\usepackage{graphicx}
\usepackage{eso-pic}
\usepackage{transparent}
\usepackage{xcolor}
\definecolor{AgencyRed}{HTML}{8B0000}
\definecolor{AgencyBlack}{HTML}{1A1A1A}

\usepackage{fancyhdr}
\pagestyle{fancy}
\fancyhf{}
\fancyhead[L]{\textbf{\thepage}}  %
\fancyhead[C]{\texttt{\textbf{\textcolor{AgencyRed}{TOP SECRET}}}} %
\fancyhead[R]{\texttt{Case File: London-Airbnb}} %

\fancyfoot[L]{} %
\fancyfoot[C]{} %
\fancyfoot[R]{\tiny \texttt{Generated by TNE$^2$+I Intelligence Unit}} %
\KOMAoption{captions}{tableheading}
\makeatletter
\@ifpackageloaded{tcolorbox}{}{\usepackage[skins,breakable]{tcolorbox}}
\@ifpackageloaded{fontawesome5}{}{\usepackage{fontawesome5}}
\definecolor{quarto-callout-color}{HTML}{909090}
\definecolor{quarto-callout-note-color}{HTML}{0758E5}
\definecolor{quarto-callout-important-color}{HTML}{CC1914}
\definecolor{quarto-callout-warning-color}{HTML}{EB9113}
\definecolor{quarto-callout-tip-color}{HTML}{00A047}
\definecolor{quarto-callout-caution-color}{HTML}{FC5300}
\definecolor{quarto-callout-color-frame}{HTML}{acacac}
\definecolor{quarto-callout-note-color-frame}{HTML}{4582ec}
\definecolor{quarto-callout-important-color-frame}{HTML}{d9534f}
\definecolor{quarto-callout-warning-color-frame}{HTML}{f0ad4e}
\definecolor{quarto-callout-tip-color-frame}{HTML}{02b875}
\definecolor{quarto-callout-caution-color-frame}{HTML}{fd7e14}
\makeatother
\makeatletter
\@ifpackageloaded{caption}{}{\usepackage{caption}}
\AtBeginDocument{%
\ifdefined\contentsname
  \renewcommand*\contentsname{Table of contents}
\else
  \newcommand\contentsname{Table of contents}
\fi
\ifdefined\listfigurename
  \renewcommand*\listfigurename{List of Figures}
\else
  \newcommand\listfigurename{List of Figures}
\fi
\ifdefined\listtablename
  \renewcommand*\listtablename{List of Tables}
\else
  \newcommand\listtablename{List of Tables}
\fi
\ifdefined\figurename
  \renewcommand*\figurename{Figure}
\else
  \newcommand\figurename{Figure}
\fi
\ifdefined\tablename
  \renewcommand*\tablename{Table}
\else
  \newcommand\tablename{Table}
\fi
}
\@ifpackageloaded{float}{}{\usepackage{float}}
\floatstyle{ruled}
\@ifundefined{c@chapter}{\newfloat{codelisting}{h}{lop}}{\newfloat{codelisting}{h}{lop}[chapter]}
\floatname{codelisting}{Listing}
\newcommand*\listoflistings{\listof{codelisting}{List of Listings}}
\makeatother
\makeatletter
\makeatother
\makeatletter
\@ifpackageloaded{caption}{}{\usepackage{caption}}
\@ifpackageloaded{subcaption}{}{\usepackage{subcaption}}
\makeatother
\usepackage{bookmark}
\IfFileExists{xurl.sty}{\usepackage{xurl}}{} % add URL line breaks if available
\urlstyle{same}
\hypersetup{
  colorlinks=true,
  linkcolor={blue},
  filecolor={Maroon},
  citecolor={Blue},
  urlcolor={Blue},
  pdfcreator={LaTeX via pandoc}}


\author{}
\date{}
\begin{document}


\begin{titlepage}
    % 使用 AddToShipoutPictureBG* 
    \AddToShipoutPictureBG*{%
        \AtPageLowerLeft{%
            \includegraphics[width=\paperwidth,height=\paperheight]{cover.png}%
        }%
    }
    \mbox{}
\end{titlepage}
\newpage

\newpage
\vspace*{1cm}

\begin{center}
\renewcommand{\arraystretch}{1.5} 
\setlength{\arrayrulewidth}{1pt}  

\begin{tabular}{|p{0.9\textwidth}|}
    \hline
    \textbf{\texttt{>> SECTION 1: INTEGRITY PLEDGE}} \\
    \hline
    \small
    \textit{We, "TNE$^2$ + I", pledge our honour that the work presented in this assessment is our own. Where information has been derived from other sources, we confirm that this has been indicated in the work. AI contributions have been clearly documented.} \\
    \\
    \textbf{TIMESTAMP:} \today \\
    \hline
    
    \textbf{\texttt{>> SECTION 2: TNE$^2$ + I (PERSONNEL)}} \\
    \hline
    \vspace{0.2cm}
    \begin{tabular}{p{0.6\textwidth} l}
        \texttt{Nadia Jimena Cabrera Salazar} & \textbf{[ID: 25198678]} \\
        \texttt{Emily Dugmore} & \textbf{[ID: 25060351]} \\
        \texttt{Julissa Tabata Paredes Condor} & \textbf{[ID: 25153376]} \\
        \texttt{Jiaying LI} & \textbf{[ID: 25211516]} \\
        \texttt{Emily Deeb} & \textbf{[ID: 25127081]} \\
    \end{tabular}
    \vspace{0.2cm}
    \\
    \hline

    \textbf{\texttt{>> SECTION 3: INTELLIGENCE REQUEST (FEEDBACK)}} \\
    \hline
    \vspace{0.1cm}
    \textbf{HQ is requested to provide critical analysis on the following sectors:}
    \begin{itemize}
    \item \textbf{Methodology Verification (Profit Loss Model)}: Our core analysis simulates the financial impact of the council tax proposal (Figure 3.3). We assumed that council tax is a valid proxy for operating costs. Is this simplification robust for a policy briefing, or are there standard models for this that we should have considered?

    \item \textbf{Visual Decryption (Landlord Concentration)}: To illustrate the dominance of professional landlords, we mapped the top 6 operators (Figure 2.3). Is this multi-plot layout effective in conveying market concentration, or would a single, aggregated heatmap be more impactful for a mayoral audience?

    \item \textbf{Narrative Logic (Policy Recommendation)}: Our central argument is that the proposed flat tax increase is a blunt instrument that disproportionately affects smaller operators in outer boroughs. Does our narrative, particularly the summaries for Figures 3.1-3.3, build a convincing case for this conclusion?
    \end{itemize}
    \vspace{0.2cm}
    \\ \hline
\end{tabular}
\end{center}

\vspace{1cm}
\begin{center}
    \textbf{\texttt{*** END OF DOSSIER HEADER ***}}
\end{center}
\newpage

\newpage{}

\newpage{}

\section{\texorpdfstring{\textbf{INTRODUCTION}}{INTRODUCTION}}\label{introduction}

In the last few years, several cities across the world have experienced
an explosive growth in Airbnb. What started as a shared economy platform
for informal home-sharing has recently transformed into a major
short-term rental market which fills the gap between traditional
residential rental housing and hotel accommodation (Quattrone \emph{et
al.} (2016)). While Airbnb advocates argue that the platform brings
extra incomes to its users and new economic activities to cities, many
raise concerns about its impact on housing markets through loss of
housing supply, increased rent and gentrification (Wachsmuth, Kerrigan
and Chaney (2017)).

This report is motivated by claims that Airbnb is `out of control' in
London, and specifically addresses the recent proposal to force all
professional landlords operating on the platform to register their
properties and face higher Council Tax rates. Using comprehensive
empirical analysis, we aim to answer five questions:

\begin{enumerate}
\def\labelenumi{\arabic{enumi}.}
\tightlist
\item
  Is Airbnb `out of control' in London?\\
\item
  How many professional landlords are there?\\
\item
  How many properties would be affected by the proposal?\\
\item
  What are the likely pros and cons of the proposal (for the Mayor,
  residents, and the city)?\\
\item
  Is it possible that Airbnb is contributing positively to social
  mobility or housing opportunities?
\end{enumerate}

\paragraph{\texorpdfstring{\textbf{Data and
methodologies}}{Data and methodologies}}\label{data-and-methodologies}

Our investigation uses listing data collected by Inside Airbnb -- an
independent, non-commercial website that scrapes publicly available
information from the Airbnb platform (Airbnb (2015)). The IA dataset
offers a more ``transparent'' view of Airbnb activity, and is commonly
used in research. However, it is not without serious limitations which
need consideration.

Importantly, the dataset acts as a snapshot of listing information as it
appeared on the day of scraping, making much of the data unreliable
because hosts could modify listing information at any moment. In 2015,
Airbnb stopped disclosing the difference between nights that are booked
by guests and nights that are `blacked-out' by hosts, making it
impossible to precisely measure occupancy and revenue. We use the
estimated occupancy and revenue generated from the IA occupancy model
(Wachsmuth, Kerrigan and Chaney (2017)). The coordinates for listings in
the IA dataset are shifted 150m in a random direction from the actual
address of the listing. It is important to note that listings which fall
close to borough boundaries may have been shifted into a different
borough, slightly affecting the precision of the results.

To determine the impact of Airbnb on the housing market, we use average
monthly rents in the private rental market (July 2023 - June 204).
Council Tax data from 2024 is used to investigate the potential impact
of introducing higher Council Tax rates to professional landlords. Both
datasets are aggregated at the borough level -- a spatial unit which
also acts as administrative boundaries -- making our results relevant
for Local Planning Authorities and actionable for local policies.

\newpage{}

\begin{tcolorbox}[enhanced jigsaw, leftrule=.75mm, rightrule=.15mm, bottomtitle=1mm, breakable, colback=white, title=\textcolor{quarto-callout-tip-color}{\faLightbulb}\hspace{0.5em}{\textbf{Assumptions}}, bottomrule=.15mm, opacityback=0, arc=.35mm, toptitle=1mm, colframe=quarto-callout-tip-color-frame, left=2mm, coltitle=black, toprule=.15mm, titlerule=0mm, opacitybacktitle=0.6, colbacktitle=quarto-callout-tip-color!10!white]

Other limitations that might need discussing (depending on how many
words we have at the end):

\begin{itemize}
\item
  Although each Airbnb listing is specified on the public Airbnb website
  with exact latitude and longitude coordinates, these coordinates have
  been shifted from the real location by up to 150m in a random
  direction (in order to protect hosts' privacy). This randomness means
  that maps which show the exact locations of listings (or rely on these
  locations for their analyses) are misleading inasmuch as they
  exaggerate the precision of the underlying spatial data.
\item
  NYC paper
\end{itemize}

\end{tcolorbox}

\section{\texorpdfstring{\textbf{Question 01: Is Airbnb out of control
in
London?}}{Question 01: Is Airbnb out of control in London?}}\label{question-01-is-airbnb-out-of-control-in-london}

Assessing whether Airbnb is ``out of control'' in London requires
defining what this means. In this analysis, Airbnb becomes out of
control when short-term rentals grow large enough to \textbf{(1) reduce
the supply of long-term housing, (2) generate strong financial
incentives to convert homes away from residential use, and (3) exceed
the city's regulatory capacity---particularly the 90-night annual limit
for short-term lets.}

Based on the analysis, Airbnb does not appear uniformly ``out of
control'' across London, but its impact is clearly concentrated, uneven,
and significant in key areas of the city, particularly in Inner London
and in larger family-sized homes.

\paragraph{\texorpdfstring{\textbf{Growth tendency in airbnb full-time
short-term
rentals}}{Growth tendency in airbnb full-time short-term rentals}}\label{growth-tendency-in-airbnb-full-time-short-term-rentals}

The dataset does not include the exact date when a property was first
listed on Airbnb. However, it does contain the date of the first review.
Because hosts typically receive their first review shortly after a guest
stays for the first time, \textbf{the year of the first review is
generally a good proxy for the year the listing became active.}

For this reason, we use the year of the first review as an estimate of
the listing's ``introduction year.'' Plotting a histogram of these years
allows us to visualize how many new Airbnb listings entered the market
each year and to observe the growth of the platform over time.

The histogram shows a clear surge in the number of new listings from
2021 onwards, quickly recovering from the Covid-19 pandemic. The strong
growth in recent years suggests that Airbnb supply has been accelerating
rather than stabilizing.

\newpage{}

\subsection{\texorpdfstring{\textbf{Figure 1.1} \textbar{} Growth of
Airbnb
Listings}{Figure 1.1 \textbar{} Growth of Airbnb Listings}}\label{figure-1.1-growth-of-airbnb-listings}

\pandocbounded{\includegraphics[keepaspectratio]{Group_Work_files/figure-pdf/cell-23-output-1.pdf}}

\textbf{Summary Figure 1.1}

\begin{itemize}
\tightlist
\item
  According to the London Plan Annual Monitoring Report 2022--23
  (Authority (2024)), annual housing completions in London have shown
  relatively limited variation in recent years. Between 2018 and 2023,
  net housing completions averaged approximately 36,600 homes per year,
  with fluctuations of only a few thousand units across the period. This
  contrasts sharply with the rapid growth of Airbnb supply. In 2023,
  Airbnb added around 5000 new listings, and in 2024 this figure rose to
  more than 7,200, representing the equivalent of 13.8\% to 20\% of
  London's yearly housing production. These proportions highlight that
  the rapid expansion of short-term rentals is effectively undermining
  the impact of London's efforts to increase housing supply. Even as
  thousands of new homes are completed each year, a significant share
  appears to be absorbed into the Airbnb market rather than contributing
  to long-term housing availability.
\end{itemize}

\newpage{}

\subsection{\texorpdfstring{\textbf{Figure 1.2} \textbar{} Growth of
Airbnb
Listings}{Figure 1.2 \textbar{} Growth of Airbnb Listings}}\label{figure-1.2-growth-of-airbnb-listings}

\pandocbounded{\includegraphics[keepaspectratio]{Group_Work_files/figure-pdf/cell-25-output-1.pdf}}

\textbf{Summary Figure 1.2}

\begin{itemize}
\tightlist
\item
  Conclusions from the map The map reveals a highly asymmetric spatial
  distribution of illegal Airbnb Activity in London. In contrast, most
  MSOAs in Outer London, show minimun levels of illegal Airbnb Activity.
  This suggests that Airbnb is not ``out of control'' everywhere, but is
  concentrated in central, high demand areas. \textbf{(This parragraph
  needs more work)}\\
\item
  Some of the darkest MSOAs---particularly in the City of London and a
  few central districts---reflect very high percentages of full-time
  Airbnb listings relative to local households.\\
\item
  This does not necessarily indicate unusually high Airbnb activity;
  rather, these areas have very small residential populations and low
  household counts. These values should therefore be interpreted with
  caution.
\end{itemize}

\newpage{}

\subsection{\texorpdfstring{\textbf{Figure 1.3} \textbar{} Scatterplot
Airbnb vs Private Rent Monthly revenue by bedroom
category}{Figure 1.3 \textbar{} Scatterplot Airbnb vs Private Rent Monthly revenue by bedroom category}}\label{figure-1.3-scatterplot-airbnb-vs-private-rent-monthly-revenue-by-bedroom-category}

\begin{center}
\pandocbounded{\includegraphics[keepaspectratio]{Group_Work_files/figure-pdf/cell-30-output-1.pdf}}
\end{center}

\textbf{Summary Figure 1.3}

\begin{itemize}
\tightlist
\item
  \textbf{Small Units (Studio/1-Bed)}: Returns are roughly equal between
  Airbnb and private rent. The points cluster along the diagonal line.
\end{itemize}

~

\begin{itemize}
\tightlist
\item
  \textbf{Large Units (2-4+ Beds)}: The gap widens significantly. Airbnb
  revenue is much higher than private rent.
\end{itemize}

~

\begin{itemize}
\tightlist
\item
  \textbf{Implication}: This creates a strong financial incentive to
  convert family-sized homes into short-term rentals, potentially
  hurting local housing supply.
\end{itemize}

\subsection{\texorpdfstring{\textbf{Figure 1.4} \textbar{} Rent Gap
(Airbnb vs Private Rent) Two Bedroom
Units}{Figure 1.4 \textbar{} Rent Gap (Airbnb vs Private Rent) Two Bedroom Units}}\label{figure-1.4-rent-gap-airbnb-vs-private-rent-two-bedroom-units}

\pandocbounded{\includegraphics[keepaspectratio]{Group_Work_files/figure-pdf/cell-32-output-1.pdf}}

\textbf{Summary Figure 1.4}

\begin{itemize}
\tightlist
\item
  This map shows the spatial distribution of the rent gap---defined as
  the ratio between Airbnb monthly revenue and private rent---for
  two-bedroom units across London boroughs. The pattern is concentrated
  in central and high-demand boroughs, suggesting greater market
  pressure on family-sized housing in these areas.\\
\item
  Boroughs are very large areas, and the rent gap can change a lot
  inside the same borough. So this map shows the general pattern, but
  not the smaller local differences.
\end{itemize}

\newpage{}

\subsection{\texorpdfstring{\textbf{Figure 1.5} \textbar{} Rent gap in
Camden, by bedroom
category}{Figure 1.5 \textbar{} Rent gap in Camden, by bedroom category}}\label{figure-1.5-rent-gap-in-camden-by-bedroom-category}

\pandocbounded{\includegraphics[keepaspectratio]{Group_Work_files/figure-pdf/cell-33-output-1.pdf}}

\textbf{Summary Figure 1.5}

\begin{itemize}
\tightlist
\item
  In a high demand , center located neighborhood like Camden, Airbnb
  earns more than private rent across all unit sizes.\\
\item
  For two-bedroom homes, Airbnb brings in around £1,000 more per month
  (about +35\%).\\
\item
  For three-bedroom homes, the gap is even larger, Airbnb earns roughly
  £1,800 extra per month (around +60\%).
\end{itemize}

Under London's regulatory framework, a property may not be rented out
for more than 90 nights a year as short-term rental without applying for
a change of use, from residential to temporary acommodation.

~

We define short-term rental listings as those which allow a minimum
length of stay shorter than 30 nights.

~

\newpage{}

Although the Inside Airbnb dataset provides calendar data showing the
availability of a listing for the next 365 days, it is impossible to
assertain whether unavailable nights are as a result of nights being
booked out by a guest or nights being `blacked-out' by the owner. Thus,
information on the number of days that listings are being rented out
annually is sparse and unreliable. Instead, the following calculation
was used to estimate the number of nights that each listing was occupied
in the last 12 months: ~

\[
O = \frac{R}{r} \times N_{min}
\]

\textbf{Where:}

\begin{itemize}
\tightlist
\item
  \(O\): Estimated Occupancy Rate (last 12 months)
\item
  \(R\): Total Number of Reviews (last 12 months)
\item
  \(r\): Review Rate (assumed percentage of guests who leave a review,
  typically 50\%)
\item
  \(N_{min}\): Average Minimum Nights per stay
\end{itemize}

This formula assumes a standard review conversion rate to estimate total
bookings from visible reviews.

\paragraph{\texorpdfstring{\textbf{Illegal
activity}}{Illegal activity}}\label{illegal-activity}

Under London's current regulations, a residential property may not be
rented out for more than 90 nights per year as a short-term rental
without first obtaining planning permission for a ``material change of
use'' from residential to temporary accommodation. We define illegal
listings as those which allow short-term rentals (rentals of less than
30 days) and exceed this 90 day limit.

Highly conservative estimates show that one in every \textbf{xxx}
listings is involved in illegal Airbnb activity, with a minimum of £16
million in illegal revenue earned in the last year. More interestingly,
\textbf{xxx}

50\% of the illegal revenue is attributable to only 171 individual hosts
(5\% of illegal hosts). Illegal Airbnb activity is mainly localised in
the city centre, with 80\% of illegal activity occurring in only nine
boroughs, all of which are located in the city centre. Although illegal
activity generates a significant amount of revenue (and this is likely
to be much higher), it is concentrated among a small number of hosts and
locations, meaning the overall impact across the city is limited.

\newpage{}

\section{\texorpdfstring{\textbf{Question 2: How many professional
landlords are
there?}}{Question 2: How many professional landlords are there?}}\label{question-2-how-many-professional-landlords-are-there}

Following the literature ((2022)), a professional landlord can be
defined by making these three assumptions:

\begin{itemize}
\tightlist
\item
  A professional landlord rents entire homes or apartments.\\
\item
  A professional landlord owns listings with a high availability per
  year.\\
\item
  A professional landlord owns more that 1 listings, since managing 2 or
  more listings requires time and coordination, which makes it unlikely
  for this to be just a ``side-job''.
\end{itemize}

The filter for landlords renting entire homes/apartments and with high
availability, so the next step if to filter multi-listers hosts to get
the final count of \textbf{professional landlords}.

\subsection{\texorpdfstring{\textbf{Figure 2.1} \textbar{} Histogram of
number of listings per professional
host}{Figure 2.1 \textbar{} Histogram of number of listings per professional host}}\label{figure-2.1-histogram-of-number-of-listings-per-professional-host}

\pandocbounded{\includegraphics[keepaspectratio]{Group_Work_files/figure-pdf/cell-40-output-1.pdf}}

~

~

~

According to our assumptions, there are \textbf{15 050 professional
hosts}, and it looks like most of them have less than 100 listings. The
hosts with more listings are excepmtions, but it's still important to
include them in the analysis since these cases are a signal that the
system is, as mentioned before, ``out of control''.

\subsection{\texorpdfstring{\textbf{Figure 2.2} \textbar{} Distribution
of total estimated revenue of professional
hosts}{Figure 2.2 \textbar{} Distribution of total estimated revenue of professional hosts}}\label{figure-2.2-distribution-of-total-estimated-revenue-of-professional-hosts}

\pandocbounded{\includegraphics[keepaspectratio]{Group_Work_files/figure-pdf/cell-43-output-1.pdf}}

~

~

~

\textbf{Following from the Output:}

Knowing which ones are the professional landlords would help deepen into
finding how do they operate in the system. A way to do this is by:

\begin{itemize}
\tightlist
\item
  Calculating the estimated revenue per host in a year\\
\item
  Plotting the spatial distribution of the top 6 professional hosts by
  number of listings\\
\item
  Finding if the number of listings per professional host is correlated
  with being a ``superhost''(Airbnb Help center (2025)), a sign of
  high-quality service
\end{itemize}

~

~

These outcomes help reveal whether their activity is driven by
commercial and profit-oriented practices.

Requirements to be a Superhost according to Article:

\begin{itemize}
\tightlist
\item
  Experience: At least 10 completed stays (or three stays totalling 100+
  nights) which is a sign of consistency\\
\item
  Responsive: Responds quickly to messages (90\% response rate)
  -\textgreater{} a sign of commitment\\
\item
  Three-month criteria: To qualify, these criteria are checked every
  three months
\end{itemize}

\newpage{}

\subsection{\texorpdfstring{\textbf{Figure 2.3} \textbar{} Spatial
distribution of the listings of the top6 professional landlords with the
most
listings}{Figure 2.3 \textbar{} Spatial distribution of the listings of the top6 professional landlords with the most listings}}\label{figure-2.3-spatial-distribution-of-the-listings-of-the-top6-professional-landlords-with-the-most-listings}

\pandocbounded{\includegraphics[keepaspectratio]{Group_Work_files/figure-pdf/cell-44-output-1.pdf}}

\newpage{}

\paragraph{\texorpdfstring{\textbf{Figure 2.4} \textbar{} Correlation
between the superhost condition and the number of
listings}{Figure 2.4 \textbar{} Correlation between the superhost condition and the number of listings}}\label{figure-2.4-correlation-between-the-superhost-condition-and-the-number-of-listings}

\pandocbounded{\includegraphics[keepaspectratio]{Group_Work_files/figure-pdf/cell-45-output-1.pdf}}

~ ~

\begin{tcolorbox}[enhanced jigsaw, leftrule=.75mm, rightrule=.15mm, bottomtitle=1mm, breakable, colback=white, title={\textbf{INTELLIGENCE BRIEF: STATISTICAL CORRELATION}}, bottomrule=.15mm, opacityback=0, arc=.35mm, toptitle=1mm, colframe=quarto-callout-important-color-frame, left=2mm, coltitle=black, toprule=.15mm, titlerule=0mm, opacitybacktitle=0.6, colbacktitle=quarto-callout-important-color!10!white]

\textbf{TARGET}: Correlation between \emph{Superhost Status} and
\emph{Listings Count}.

\begin{itemize}
\tightlist
\item
  \textbf{Pearson Coefficient (\(r\))}: \textbf{-0.039}
\item
  \textbf{Significance Level (\(p\))}: \textbf{p \textless{} 0.05}
\end{itemize}

\textbf{TACTICAL ASSESSMENT}: The correlation is negligible (close to
zero), suggesting Superhost status is \textbf{independent} of portfolio
size. However, the result is statistically significant (\(p < 0.05\)),
confirming this pattern is not random noise.

\end{tcolorbox}

~

~

\begin{itemize}
\item
  The plots show that the top 6 professional landlords are spread across
  London, with some clustered in central areas and high-value boroughs.
  This spatial distribution, along with the portfolio size and total
  estimated revenue for all the listings, indicates that some of these
  hosts are actually commercial operators, not just individuals sharing
  their homes as Airbnb was initially intended.
\item
  In simple terms, our analysis shows that having many Airbnb listings
  does not mean a host provides better service. This means that having
  an extensive portfolio does not necessarily indicate higher-quality
  service, and that commercial, very profitable listings drive the
  number of listings, as the analysis shows.
\end{itemize}

\newpage{}

\section{\texorpdfstring{\textbf{Question 3: How many properties would
be affected by the opposition's
proposal?}}{Question 3: How many properties would be affected by the opposition's proposal?}}\label{question-3-how-many-properties-would-be-affected-by-the-oppositions-proposal}

This question looks at \textbf{professional full-time entire-home Airbnb
listings only}, attaches each listing to its borough's and accordingly
the average council tax, in order evaluate several listings affected by
the oppsional propusal of increasing the council tax, we are applting a
\textbf{40\% council tax increase} as a stndard value of growth (140\%)
to estimate how many properties see a minor 10\% drop in profit.

\textbf{The workflow:} clean and merge council tax with borough
polygons, use filtered listings, spatially join listings to boroughs,
compare average annual revenue vs average council tax by borough, then
simulate a 40\% tax rise and calculate profit loss per listing before
aggregating to borough level.

\subsection{\texorpdfstring{\textbf{Figure 3.1} \textbar{} Average
Airbnb Revenue vs Average Council Tax per
Borough}{Figure 3.1 \textbar{} Average Airbnb Revenue vs Average Council Tax per Borough}}\label{figure-3.1-average-airbnb-revenue-vs-average-council-tax-per-borough}

\pandocbounded{\includegraphics[keepaspectratio]{Group_Work_files/figure-pdf/cell-54-output-1.pdf}}

~

\textbf{Summary Figure 3.1}

\begin{itemize}
\tightlist
\item
  Average Airbnb revenue is much higher than council tax (≈ £28k vs ≈
  £1.9k). The map shows an apparent mismatch: the highest-earning
  boroughs (Westminster, Kensington \& Chelsea, City of London) pay some
  of the lowest council taxes, while outer boroughs with weak Airbnb
  income face higher council tax and very few listings. In short, Airbnb
  profitability is concentrated in the centre, and council tax has
  almost no bite where the profits are highest.
\end{itemize}

\newpage{}

\subsection{\texorpdfstring{\textbf{Figure 3.2} \textbar{} Linear
Relationship Between Revenue and Profit
Loss}{Figure 3.2 \textbar{} Linear Relationship Between Revenue and Profit Loss}}\label{figure-3.2-linear-relationship-between-revenue-and-profit-loss}

\pandocbounded{\includegraphics[keepaspectratio]{Group_Work_files/figure-pdf/cell-59-output-1.pdf}}

~

~

\begin{tcolorbox}[enhanced jigsaw, leftrule=.75mm, rightrule=.15mm, bottomtitle=1mm, breakable, colback=white, title={INTELLIGENCE ANALYSIS: REVENUE MODEL}, bottomrule=.15mm, opacityback=0, arc=.35mm, toptitle=1mm, colframe=quarto-callout-important-color-frame, left=2mm, coltitle=black, toprule=.15mm, titlerule=0mm, opacitybacktitle=0.6, colbacktitle=quarto-callout-important-color!10!white]

\textbf{DATA INTERCEPTION}: This regression is derived from a filtered
sample of \textbf{8,760} listings.

\textbf{DERIVED FORMULA}:

\[
\text{profit\_loss\_pct} \approx -0.0000 + 0.05706396 \times \text{annual\_revenue}
\]

\end{tcolorbox}

~

~

\textbf{Summary Figure 3.2}

\begin{itemize}
\tightlist
\item
  Low-earning listings get hit with the opposition proposal. And
  high-earning listings aren't affected. When revenue is small, a 40\%
  tax increase can wipe out a big chunk of profit --- sometimes almost
  all of it. But once revenue rises, council tax becomes tiny in
  comparison, so the profit loss drops close to zero.
\end{itemize}

\newpage{}

\subsection{\texorpdfstring{\textbf{Figure 3.3} \textbar{} Map of
Average Profit Loss per
Borough}{Figure 3.3 \textbar{} Map of Average Profit Loss per Borough}}\label{figure-3.3-map-of-average-profit-loss-per-borough}

~

~

\begin{table}[ht]
\centering
\textbf{Average Profit Loss per Borough}\\[1.5em]
\small
\begin{tabular}{lrrr}
\toprule
Borough & Count (Pro) & Avg Loss (\%) & Med Loss (\%) \\
\midrule
havering & 17 & 9.7 & 7.8 \\
croydon & 99 & 9.2 & 7.2 \\
sutton & 25 & 8.5 & 6.4 \\
bexley & 18 & 7.4 & 6.8 \\
harrow & 39 & 7.1 & 6.4 \\
hillingdon & 30 & 7.0 & 5.4 \\
redbridge & 38 & 6.4 & 4.7 \\
waltham forest & 111 & 6.2 & 4.7 \\
enfield & 39 & 6.0 & 5.2 \\
kingston upon thames & 32 & 6.0 & 5.1 \\
brent & 248 & 5.7 & 4.5 \\
ealing & 182 & 5.4 & 4.5 \\
lewisham & 117 & 5.4 & 4.6 \\
haringey & 162 & 5.4 & 4.4 \\
hounslow & 90 & 5.3 & 3.4 \\
barnet & 223 & 5.1 & 4.2 \\
merton & 46 & 5.0 & 4.0 \\
bromley & 46 & 4.9 & 4.2 \\
barking and dagenham & 26 & 4.8 & 4.8 \\
richmond upon thames & 76 & 4.8 & 4.2 \\
hackney & 441 & 4.1 & 3.4 \\
lambeth & 380 & 4.0 & 3.1 \\
islington & 525 & 3.8 & 2.9 \\
greenwich & 126 & 3.7 & 3.2 \\
newham & 167 & 3.7 & 3.2 \\
southwark & 409 & 3.3 & 2.5 \\
tower hamlets & 607 & 3.0 & 2.6 \\
camden & 843 & 3.0 & 2.5 \\
hammersmith and fulham & 467 & 2.6 & 2.2 \\
kensington and chelsea & 936 & 2.0 & 1.7 \\
wandsworth & 292 & 1.9 & 1.6 \\
city and county of the city of london & 81 & 1.5 & 1.2 \\
city of westminster & 1911 & 1.2 & 0.9 \\
\bottomrule
\end{tabular}

\vspace{1ex}
\begin{minipage}{0.85\textwidth}
\raggedright
\footnotesize
\textit{Note:} \\
\textbf{Count (Pro)}: Number of professional full-time listings; \\
\textbf{Avg/Med Loss (\%)}: Average and Median percentage of profit lost under the new tax policy.
\end{minipage}
\end{table}

\pandocbounded{\includegraphics[keepaspectratio]{Group_Work_files/figure-pdf/cell-62-output-1.pdf}}

\textbf{Summary Figure 3.3}

\begin{itemize}
\tightlist
\item
  We can see that the average profit loss per borough ranges from abou
  \textbf{1--5\% in central boroughs} to 10\% in some outer boroughs,
  meaning outer borough hosts are more affected by the opposition's
  suggestion.
\end{itemize}

~

~

\textbf{Short General Conclusion on discussion 3:}

\begin{itemize}
\tightlist
\item
  Only a small share of professional landlords are meaningfully affected
  by a 40\% council tax increase:\textbf{272 out of 8,849} listings
  (about 3\%) lose more than 10\% of their profit. The impact is
  concentrated in outer London and lower-revenue boroughs, while
  high-revenue central boroughs show almost no effect. This suggests
  that a flat council tax rise is not a strong lever for changing
  professional Airbnb behaviour; more targeted measures based on revenue
  or the number of listings would be needed to shift the sector in any
  meaningful way.
\end{itemize}

\newpage{}

\section{\texorpdfstring{\textbf{Policy Analysis: Regulating Short-Term
Lets for Housing
Equity}}{Policy Analysis: Regulating Short-Term Lets for Housing Equity}}\label{policy-analysis-regulating-short-term-lets-for-housing-equity}

\begin{enumerate}
\def\labelenumi{\arabic{enumi}.}
\tightlist
\item
  \textbf{The Context: The Scale of the Problem}\\
  Our analysis shows that Airbnb contributes to London's housing
  shortage. London adds about \textbf{35,000 homes per year} (Authority
  (2024)), \textbf{while Airbnb introduces 5,000--7,200 new listings
  annually (Q1)}. As a result, \textbf{14--20\%} of new housing is
  absorbed by the short-term rental market.
\end{enumerate}

~

\begin{enumerate}
\def\labelenumi{\arabic{enumi}.}
\setcounter{enumi}{1}
\tightlist
\item
  \textbf{Critique of the Opposition's Proposal}\\
  The opposition proposes raising council tax for Airbnb hosts and
  ``professional landlords.'' While this addresses the issue, our
  analysis data shows this approach is \textbf{blunt and
  inconsistent}.\\
\end{enumerate}

\begin{itemize}
\tightlist
\item
  ``Professional landlords'' as such, with multiple listings, especially
  in central boroughs such as Westminster and Kensington \& Chelsea,
  achieve higher profits. Even with a suggested \textbf{40\% council tax
  increase (Q3)}, most high-revenue listings remain profitable.\\
\item
  This ``minimal'' tax disproportionately impacts lower-revenue, casual
  hosts in outer boroughs who cannot absorb the cost, while failing to
  deter well-capitalised operators who most affect housing supply.
\end{itemize}

~

\begin{enumerate}
\def\labelenumi{\arabic{enumi}.}
\setcounter{enumi}{2}
\tightlist
\item
  \textbf{The Better Alternative: Comprehensive Regulation}\\
  Rather than relying solely on council tax, the Mayor should support
  the opposition's intent while adopting a different approach. Effective
  policy that can address the market's underlying mechanisms:\\
\end{enumerate}

\begin{itemize}
\tightlist
\item
  \textbf{Registration and Enforcement}: Require strict host
  registration and impose penalties on non-compliant platforms. Primary
  Residence Rules: Adopt international standards, such as \textbf{New
  York's Local Law 18} (Mayor's Office of Special Enforcement (OSE)
  (2022)) and \textbf{Amsterdam's 30-night cap} (Gemeente (2025)), that
  require hosts to be primary residents.
\item
  \textbf{Smart Taxation}: Replace flat council tax increases with a
  combined \textbf{revenue-based and volume-based tax system} that
  targets high-yield portfolios (Q2).
\item
  \textbf{Supply Protection}: Ban short-term letting of council housing
  and newly built units.
\end{itemize}

~

\begin{enumerate}
\def\labelenumi{\arabic{enumi}.}
\setcounter{enumi}{3}
\tightlist
\item
  \textbf{Reframing the Narrative: Social Mobility and Opportunity}\\
  Regulating Airbnb should be seen not just as a tax issue, but as a
  positive step for \textbf{social mobility and housing opportunity}.
  London Councils (2025) ((2025)) says that 24\% of renters feel secure
  about continuing to live in the city, and 1 in 4 of them thinks they
  may need to leave the city to find more affordable rent in the next 12
  months. In addition, Councillor Claire Holland claims:
\end{enumerate}

\begin{quote}
\emph{``London faces the most severe homelessness emergency in the
country. Driven by the worsening shortage of affordable housing, far too
many Londoners are struggling with their housing costs and at risk of
becoming homeless''.}
\end{quote}

\newpage{}

~ ~ ~

\begin{center}
\Large\bfseries
Pros and Cons of the Council Tax Proposal
\end{center}

~ ~ ~ ~ ~ ~ ~

\vspace{0.5cm}
\begin{center}
\renewcommand{\arraystretch}{1.4}
\setlength{\arrayrulewidth}{1.5pt}

\definecolor{IntelRed}{HTML}{8B0000}
\definecolor{IntelBlue}{HTML}{000080}

\begin{tabular}{|p{0.95\textwidth}|}
    \hline
    \textbf{\texttt{>> IMPACT ASSESSMENT MATRIX (CLASSIFIED)}} \\
    \hline
    
    % --- Section 1: Mayor ---
    \vspace{0.1cm}
    \textbf{\large 1. TARGET: MAYORAL OFFICE} \\
    \vspace{0.1cm}
    
    \textbf{\textcolor{IntelBlue}{[+] STRATEGIC GAINS}}
    \begin{itemize}
        \item \textbf{Narrative Control}: Mayor can claim victory in "Making Airbnb pay more".
        \item \textbf{Resource Acquisition}: Generates new revenue streams for enforcement.
    \end{itemize}
    
    \textbf{\textcolor{IntelRed}{[-] TACTICAL RISKS}}
    \begin{itemize}
        \item \textbf{Symbolic Policy}: Risk of being perceived as performative rather than practical.
        \item \textbf{Asymmetric Impact}: Penalizes small hosts in outer boroughs; wealthy operators in Westminster remain unaffected.
    \end{itemize}
    \\ \hline

    % --- Section 2: Residents ---
    \vspace{0.1cm}
    \textbf{\large 2. TARGET: CIVILIAN POPULATION} \\
    \vspace{0.1cm}
    
    \textbf{\textcolor{IntelBlue}{[+] STRATEGIC GAINS}}
    \begin{itemize}
        \item \textbf{Supply Recovery}: Marginal listings may revert to long-term housing as profits thin.
    \end{itemize}
    
    \textbf{\textcolor{IntelRed}{[-] TACTICAL RISKS}}
    \begin{itemize}
        \item \textbf{Profit Gap Persists}: Short-term rentals remain far more profitable; most landlords will absorb the tax.
    \end{itemize}
    \\ \hline
    
    % --- Section 3: City ---
    \vspace{0.1cm}
    \textbf{\large 3. TARGET: CITY INFRASTRUCTURE} \\
    \vspace{0.1cm}
    
    \textbf{\textcolor{IntelBlue}{[+] STRATEGIC GAINS}}
    \begin{itemize}
        \item \textbf{Demand Shift}: Reduces pressure on residential zones; pushes tourists back to hotels.
    \end{itemize}
    
    \textbf{\textcolor{IntelRed}{[-] TACTICAL RISKS}}
    \begin{itemize}
        \item \textbf{Misaligned Targeting}: Policy targets tax bands rather than the root drivers (Revenue \& Portfolio Size).
    \end{itemize}
    \vspace{0.1cm}
    \\ \hline
    
\end{tabular}
\end{center}

\newpage{}

Drawing on Dayne Lee's study of the Los Angeles housing market (Dayne
(2016)), this narrative shift could help see Airbnb regulation as an
opportunity to prioritize residential stability in high-demand areas.
Following this, policy measures can be justified by: - Protecting
Communities: Preventing long-term rentals from becoming short-term lets
helps safeguard against eviction and displacement. - Funding
Affordability: Revenue from high-volume landlords should be
redistributed to less privileged boroughs to support social housing
programs. - Stopping Gentrification: Reducing the profitability of
short-term lets slows neighbourhood change that displaces lower-income
households from quality amenities.

\subsection{Implications for the
Mayor}\label{implications-for-the-mayor}

In the context of the election, the opposition's proposal could be seen
as merely a response to individual misconduct, allowing the Mayor to
shift the narrative to a broader, more powerful regulation in the
collective interest. The strategy presented combines registration,
regulation, and redistribution to less privileged areas of the city.
This would show responsiveness from the current administration while
addressing a structural issue about housing instability.

~

This is key to the campaign's communication: it acknowledges public
concern and leverages it. Doing this enables the Mayor to shift the
debate from political blame to policy effectiveness, using data to
justify a proportionate and socially equitable response.

\newpage{}

\section*{References}\label{references}
\addcontentsline{toc}{section}{References}

\phantomsection\label{refs}
\begin{CSLReferences}{0}{1}
\bibitem[\citeproctext]{ref-airbnbhelpcenter}
Airbnb Help center (2025) {`Airbnb: {Holiday Rentals}, {Cabins}, {Beach
Houses}, {Unique Homes} \& {Experiences}'}, \emph{Airbnb Help center}.
https://www.airbnb.co.uk/help/article/829.

\bibitem[\citeproctext]{ref-insideairbnb}
Airbnb, I. (2015) {`Data {Assumptions}'}.
https://insideairbnb.com/data-assumptions/.

\bibitem[\citeproctext]{ref-greaterlondonauthority}
Authority, G. L. (2024) {`London {Plan Annual Monitoring Report} 19
2021/22'}.

\bibitem[\citeproctext]{ref-londoncouncils}
Councils, L. (2025) {`As the housing crisis persists, 56\% of
{Londoners} support building on grey belt'}.
https://www.londoncouncils.gov.uk/news-and-press-releases/2025/housing-crisis-persists-56-londoners-support-building-grey-belt.

\bibitem[\citeproctext]{ref-DayneLee}
Dayne, L. (2016) {`How {Airbnb Short-Term Rentals Exacerbate Los
Angeles}'s {Affordable Housing Crisis}'}.

\bibitem[\citeproctext]{ref-amsterdam}
Gemeente (2025) {`Reporting holiday rentals'}.
https://www.amsterdam.nl/en/housing/holiday-rentals/reporting-holiday-rentals/;
Gemeente Amsterdam.

\bibitem[\citeproctext]{ref-nycose}
Mayor's Office of Special Enforcement (OSE) (2022) {`Registration {Law}
- {OSE}'}.
https://www.nyc.gov/site/specialenforcement/registration-law/registration.page.

\bibitem[\citeproctext]{ref-quattrone2016}
Quattrone, G. \emph{et al.} (2016) {`Who {Benefits} from the "{Sharing}"
{Economy} of {Airbnb}?'}, in \emph{Proceedings of the 25th
{International Conference} on {World Wide Web}}. Republic; Canton of
Geneva, CHE: International World Wide Web Conferences Steering Committee
({WWW} '16), pp. 1385--1394. doi:
\href{https://doi.org/10.1145/2872427.2874815}{10.1145/2872427.2874815}.

\bibitem[\citeproctext]{ref-shabrina2022}
Shabrina, Z., Arcaute, E. and Batty, M. (2022) {`Airbnb and its
potential impact on the {London} housing market'}, \emph{Urban Studies},
59(1), pp. 197--221. doi:
\href{https://doi.org/10.1177/0042098020970865}{10.1177/0042098020970865}.

\bibitem[\citeproctext]{ref-mcgilluniversity}
Wachsmuth, D., Kerrigan, D. and Chaney, D. (2017) {`The {High Cost} of
{Short-Term Rentals} in {New York City}'}.

\end{CSLReferences}




\end{document}
